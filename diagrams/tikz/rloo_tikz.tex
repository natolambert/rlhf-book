% RLOO (REINFORCE Leave-One-Out) Architecture Diagram
% Shows: Policy, Reference, Reward models with leave-one-out baseline
% KL penalty folded into shaped reward (like PPO)
% No value network, no clipping

\documentclass{article}
\usepackage[margin=0.3cm, paperwidth=27cm, paperheight=12cm]{geometry}
\usepackage{tikz}
\usetikzlibrary{positioning, arrows.meta, shapes.geometric, fit, calc, backgrounds}
\pagestyle{empty}

\begin{document}
% Shared TikZ styles for RLHF algorithm diagrams
% Include with: % Shared TikZ styles for RLHF algorithm diagrams
% Include with: % Shared TikZ styles for RLHF algorithm diagrams
% Include with: \input{styles_rlhf.tex} BEFORE tikzpicture
% Requires: \usetikzlibrary{positioning, arrows.meta, shapes.geometric, fit, calc, backgrounds}

\tikzset{
    % Node styles
    trained/.style={
        rectangle,
        draw=black,
        fill=yellow!30,
        minimum width=1.8cm,
        minimum height=0.8cm,
        font=\footnotesize,
        rounded corners=3pt,
        align=center
    },
    frozen/.style={
        rectangle,
        draw=blue!50,
        fill=blue!12,
        minimum width=1.8cm,
        minimum height=0.8cm,
        font=\footnotesize,
        rounded corners=3pt,
        align=center
    },
    plain/.style={
        rectangle,
        draw=black!70,
        fill=white,
        minimum width=0.8cm,
        minimum height=0.6cm,
        font=\footnotesize,
        rounded corners=3pt
    },
    plainwide/.style={
        rectangle,
        draw=black!70,
        fill=white,
        minimum width=1.6cm,
        minimum height=0.8cm,
        font=\footnotesize,
        rounded corners=3pt,
        align=center
    },
    groupbox/.style={
        rectangle,
        draw=black!30,
        fill=black!5,
        rounded corners=4pt,
        inner sep=8pt
    },
    operator/.style={
        circle,
        draw=black!70,
        fill=white,
        minimum size=0.6cm,
        font=\footnotesize
    },
    seclabel/.style={
        font=\bfseries\Large
    },
    arrow/.style={
        ->,
        >=Stealth,
        thick
    },
    feedbackarrow/.style={
        ->,
        >=Stealth,
        thick,
        draw=black,
        dashed
    },
    legendbox/.style={
        rectangle,
        draw=black,
        minimum width=1.8cm,
        minimum height=0.8cm,
        font=\footnotesize,
        rounded corners=3pt,
        align=center
    },
    note/.style={
        font=\scriptsize\itshape,
        text=black!70
    }
}
 BEFORE tikzpicture
% Requires: \usetikzlibrary{positioning, arrows.meta, shapes.geometric, fit, calc, backgrounds}

\tikzset{
    % Node styles
    trained/.style={
        rectangle,
        draw=black,
        fill=yellow!30,
        minimum width=1.8cm,
        minimum height=0.8cm,
        font=\footnotesize,
        rounded corners=3pt,
        align=center
    },
    frozen/.style={
        rectangle,
        draw=blue!50,
        fill=blue!12,
        minimum width=1.8cm,
        minimum height=0.8cm,
        font=\footnotesize,
        rounded corners=3pt,
        align=center
    },
    plain/.style={
        rectangle,
        draw=black!70,
        fill=white,
        minimum width=0.8cm,
        minimum height=0.6cm,
        font=\footnotesize,
        rounded corners=3pt
    },
    plainwide/.style={
        rectangle,
        draw=black!70,
        fill=white,
        minimum width=1.6cm,
        minimum height=0.8cm,
        font=\footnotesize,
        rounded corners=3pt,
        align=center
    },
    groupbox/.style={
        rectangle,
        draw=black!30,
        fill=black!5,
        rounded corners=4pt,
        inner sep=8pt
    },
    operator/.style={
        circle,
        draw=black!70,
        fill=white,
        minimum size=0.6cm,
        font=\footnotesize
    },
    seclabel/.style={
        font=\bfseries\Large
    },
    arrow/.style={
        ->,
        >=Stealth,
        thick
    },
    feedbackarrow/.style={
        ->,
        >=Stealth,
        thick,
        draw=black,
        dashed
    },
    legendbox/.style={
        rectangle,
        draw=black,
        minimum width=1.8cm,
        minimum height=0.8cm,
        font=\footnotesize,
        rounded corners=3pt,
        align=center
    },
    note/.style={
        font=\scriptsize\itshape,
        text=black!70
    }
}
 BEFORE tikzpicture
% Requires: \usetikzlibrary{positioning, arrows.meta, shapes.geometric, fit, calc, backgrounds}

\tikzset{
    % Node styles
    trained/.style={
        rectangle,
        draw=black,
        fill=yellow!30,
        minimum width=1.8cm,
        minimum height=0.8cm,
        font=\footnotesize,
        rounded corners=3pt,
        align=center
    },
    frozen/.style={
        rectangle,
        draw=blue!50,
        fill=blue!12,
        minimum width=1.8cm,
        minimum height=0.8cm,
        font=\footnotesize,
        rounded corners=3pt,
        align=center
    },
    plain/.style={
        rectangle,
        draw=black!70,
        fill=white,
        minimum width=0.8cm,
        minimum height=0.6cm,
        font=\footnotesize,
        rounded corners=3pt
    },
    plainwide/.style={
        rectangle,
        draw=black!70,
        fill=white,
        minimum width=1.6cm,
        minimum height=0.8cm,
        font=\footnotesize,
        rounded corners=3pt,
        align=center
    },
    groupbox/.style={
        rectangle,
        draw=black!30,
        fill=black!5,
        rounded corners=4pt,
        inner sep=8pt
    },
    operator/.style={
        circle,
        draw=black!70,
        fill=white,
        minimum size=0.6cm,
        font=\footnotesize
    },
    seclabel/.style={
        font=\bfseries\Large
    },
    arrow/.style={
        ->,
        >=Stealth,
        thick
    },
    feedbackarrow/.style={
        ->,
        >=Stealth,
        thick,
        draw=black,
        dashed
    },
    legendbox/.style={
        rectangle,
        draw=black,
        minimum width=1.8cm,
        minimum height=0.8cm,
        font=\footnotesize,
        rounded corners=3pt,
        align=center
    },
    note/.style={
        font=\scriptsize\itshape,
        text=black!70
    }
}


\begin{tikzpicture}

% Input s_t (state/prompt)
\node[plain] (st) at (1.5, 0) {$s_t$};

% Legend (below input/policy area)
\node[font=\scriptsize\itshape] (leg-key) at (1.6, -1.4) {Key};
\node[legendbox, fill=blue!15, draw=blue!70] (leg-trained) at (2.6, -1.4) {Trained};
\node[legendbox, fill=red!12, draw=red!50] (leg-frozen) at (3.9, -1.4) {Frozen};
\node[draw=black!50, dotted, rounded corners=3pt, inner sep=6pt, fit=(leg-key)(leg-frozen)] {};

% Policy Model (trained)
\node[trained] (policy) at (3.5, 0) {Policy\\Model};

% Multiple outputs in group (K samples per prompt)
\node[plain] (a1) at (5.8, 0.95) {$a_1$};
\node[plain] (a2) at (5.8, 0) {$a_2$};
\node[font=\footnotesize] at (5.8, -0.6) {...};
\node[plain] (aK) at (5.8, -1.2) {$a_K$};

% Reward Model (frozen, top)
\node[frozen] (reward) at (8.2, 0.95) {Reward\\Model};

% Reference Model (frozen, bottom)
\node[frozen] (ref) at (8.2, -1.2) {Reference\\Model};

% Note: shaped reward equation (between Reward and Reference models)
\node[note] at (8.2, -0.1) {$\tilde{r} = r - \beta \cdot KL$};

% Oplus operator (moved left)
\node[operator] (oplus) at (9.5, -0.1) {$\oplus$};

% Shaped rewards (r̃ = r - β·KL)
\node[plain] (r1) at (11.2, 0.95) {$\tilde{r}_1$};
\node[plain] (r2) at (11.2, 0) {$\tilde{r}_2$};
\node[font=\footnotesize] at (11.2, -0.6) {...};
\node[plain] (rK) at (11.2, -1.2) {$\tilde{r}_K$};

% Leave-One-Out box with equation
\node[plainwide, minimum width=2.4cm, minimum height=1.0cm] (loo) at (14.0, -0.1) {$A_i = \tilde{r}_i - \frac{1}{K-1}\sum\limits_{j \neq i} \tilde{r}_j$};

% Multiple advantages
\node[plain] (A1) at (16.7, 0.95) {$A_1$};
\node[plain] (A2) at (16.7, 0) {$A_2$};
\node[font=\footnotesize] at (16.7, -0.6) {...};
\node[plain] (AK) at (16.7, -1.2) {$A_K$};

% Gray background boxes for groups
\begin{scope}[on background layer]
\node[groupbox, fit=(a1)(aK)] (abox) {};
\node[groupbox, fit=(r1)(rK)] (rbox) {};
\node[groupbox, fit=(A1)(AK)] (Abox) {};
\end{scope}

% ============ ARROWS ============

% s_t -> Policy
\draw[arrow] (st) -- (policy);

% Policy -> outputs group
\draw[arrow] (policy.east) -- (abox.west);

% Outputs -> Reward Model (single arrow)
\draw[arrow] (abox.east) to[out=0, in=180] (reward.west);

% Outputs -> Reference Model (single arrow)
\draw[arrow] (abox.east) to[out=0, in=180] (ref.west);

% Reward Model -> oplus
\draw[arrow] (reward.east) to[out=0, in=90] (oplus.north);

% Reference Model -> oplus (KL contribution)
\draw[arrow] (ref.east) to[out=0, in=-90] (oplus.south);

% Oplus -> shaped rewards
\draw[arrow] (oplus.east) -- (rbox.west);

% Shaped rewards -> LOO
\draw[arrow] (rbox.east) -- (loo.west);

% LOO -> Advantages
\draw[arrow] (loo.east) -- (Abox.west);

% Hidden waypoint for feedback arrow (above oplus area)
\coordinate (feedback-mid) at ($(oplus) + (0, 2.5)$);

% FEEDBACK: A -> Policy (REINFORCE, no clipping) - made more prominent
\draw[feedbackarrow, line width=1.2pt] (Abox.north west) to[out=120, in=0]
    (feedback-mid) to[out=180, in=0]
    ($(policy.north) + (0, 1.6)$) to[out=180, in=90]
    (policy.north);


\end{tikzpicture}

\end{document}
